L'affectation est un problème d'optmisation combinatoire qui consiste, dans sa version simple, à affecter un certain nombre de ressources disponibles à un nombres de tâches dans l'objectif d'optimiser une fonction objectif. Il peut s'agir de minimiser des coûts ou maximiser les bénéfices. Ce problème peut être résolu en temps polynomial par la méthode hongroise également appelée algorithme de Kunh.\\
On peut étendre ce problème par des objectifs multiples, des contraintes changeantes en temps réel ou encore des contraintes probabilites traduisant une observation ou une connaissance partielle de l'environnement. On peut alors parler plus largement de plannification, qui est un domaine ouvert de l'intelligence artificielle où de nombreuses formulations font apparaître des problèmes de la classe NP-difficile ou NP-complet sur lesquels de nombreuses équipes de recherches travaillent.\\\\

Dans le cadre de notre projet C++ nous avons choisi de réaliser un framework permettant la modélisation et l'implémentation de problème de plannification spaciale sous contrainte. Il s'agira de proposer, comme nous le verrons en détail par la suite, des structures de données de partitionnement spacial permettant la réduction de la complexité d'heuristiques par le principe de Divide and Conquer, divers algorithmes d'affectation selon plusieurs approches (ressources vers objectifs, objectifs vers ressources), et également divers moteurs de modélisation de contraintes et objectifs pour la plannification. Comme notre plannification concerne de l'affectation spatiale, des algorithmes de plus court chemins seront également proposés.\\
Enfin, pour illustrer la résolution temps réel des problèmes, en plus du framework, une application de simulation d'une Intelligence Artificielle sera développée. Il s'agira d'une application graphique montrant des unités d'un jeu vidéo affectées à divers postes pour maximiser divers objectifs pouvant changer au court du temps (équilibrage de ressources, objectif de construction, par exemple).\\\\

Le projet étant ambitieux à la vue de la multitude des algorithmes existants tant pour le plus court chemin que pour l'indexation spatiale voire pour la description des contraintes et objectifs, le travail réalisé sera axé sur la modélisation et l'architecture du framework. L'implémentation sera partielle mais les bonnes pratiques de développement occuperont une part importante de celle-ci dans le but de valoriser le livrable et permettre une implémentation complète en dehors du cadre de ce projet.\\\\

Ce rapport se divise en plusieurs chapitre. Le premier présentera dans un premier temps la méthodologie adoptée et les pratiques de développment. Le second chapitre traitera de la modélisation. Dans un premier temps, des concepts généraux seront exposés, puis nous exposerons l'analyse des besoins, l'identification des briques logicielles et l'architecture globale du framework. Enfin, une présentation détaillée des divers composants sera effectuée : partitionnement et indexation spatial, algorithmes de plus court chemin, algorithmes d'affectation, moteur de contraintes et plannification.\\
Un troisième chapitre sera dédié à l'implémentation et un quatrième présentera l'application de démonstration : technologies, modélisation et implémentation.
