Le projet, bien que très incomplet en terme d'implémentation a été très intéressant et enrichissant d'un point de vue génie logiciel. Malgré quelques difficultés sur la modélisation et quelques modifications à effectuer lors de l'implémentation, le projet nous a permis de renforcer une méthodologie efficace de développement.\\
On regrettera peut-être que la méthode utilisée n'ait pas été incrémentale (mais le temps dédié au projet est certainement trop court pour cela).\\\\

La différence de niveau technique en C++ n'a pas du tout été un problème et les échanges techniques ont été intéressant et enrichissant, permettant à l'un d'en apprendre plus en C++ et à l'autre membre du projet de tester de nouvelles choses en terme d'architecture et de techniques.\\\\

Le regret lié au manque de temps provient du fait que le projet était à la base calibré pour 4 personnes mais que les effectifs ont été réduit à 2 sans changer le sujet. Ceci dit, il semble que la modélisation tienne globalement la route et pose les bases d'un logiciel correct, malgré une implémentation partielle.\\
Cependant, chaque module pourrait largement être raffiné, que ce soit la gestion de l'environnement, de l'IA, des contraintes, etc. L'objectif étant plutôt d'amorcer le projet et de fournir un prototype fonctionnel de ce que pourrait être un tel framework, pour le continuer par la suite, en dehors du cours C++.
