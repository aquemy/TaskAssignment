Un système de construction a été mis en place pour le livrable grâce à CMake et les outils gravitants autour.
CMake permet de gérer l'ensemble des dépendances, générer les exécutables pour les tests, les binaires de librairies, etc.\\\\

Le système de build mis en place est principalement localisé dans le dossier cmake de la racine du projet.\\
Il comprend les fichiers suivant :
\begin{itemize}
\item Config.cmake, qui gère les différentes options de compilation
\item Files.cmake, qui liste l'ensemble des fichiers nécessaires à la création de la bibliothèque. Notamment, il liste à la fois les fichiers .cpp ne contenant que l'implémentation mais également les fichiers .hpp des classes templatées qui appellent eux même l'implémentation.
\item Macro.cmake, qui comprends les commandes personnalisées. La seule commande disponible permet d'ajouter facilement de nouvelles lessons.
\item Package.cmake, qui gère le packaging basique.
\item Target.cmake, comprenant des cibles personnalisées, notamment quelques raccourcis d'utilisation.
\end{itemize}

Le livrable n'étant pas complètement implémenté, le processus d'installation n'a pas été ajouté.
