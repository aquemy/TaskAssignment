
% -------------------------------------
% ---			core/ 			---
% -------------------------------------

% -------------------------
\subsection{controller}
% -------------------------

On va tester la fonction \texttt{run()}, pour cela on va utiliser une fonction qui nous permetrra de compter le nombre de passage ie le nombre d'étapes. Il suffira de vérifier que le compteur a atteint le nombre d'étapes prévues.


% -------------------------------------
% ---		core/continue/			---
% -------------------------------------


% -------------------------
\subsection{stepContinue}
% -------------------------

On doit tester la fonction \texttt{\_check()}. Deux cas sont a tester, lorsque l'argument \texttt{steps} $>1$ (retourne vrai) et lorsqu'il est $\leq1$ (retourne faux).\\

% -------------------------
\subsection{timeContinue}
% -------------------------


On doit tester la fonction \texttt{\_check()}.\\
Si le chronomètre n'a pas été lancé, la fonction doit le lancer (cf logger).\\
Ensuite, on testera un cas où la durée limite n'a pas été atteinte, et un autre où elle a été dépassée.\\
On testera également le cas d'erreur où la durée est négative.
