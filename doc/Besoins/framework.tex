\section{Description du système et objectifs}

Le système doit permettre pour l'utilisateur la création d'une intelligence artificielle pour résoudre le problème d'affectation.\\
La création de l'IA se fait en plusieurs étapes :
\begin{itemize}
\item Définition de l'environnement (tâches, objectifs & contraintes, unités)
\item Définition d'une méthode d'apprentissage et d'un modèle d'observation de l'environnement
\item Création d'une ou plusieurs stratégies de résolution du problème d'affectation
\end{itemize}

Une stratégie classique consiste à :
\begin{itemize}
\item Discrétiser l'espace
\item Indexer les unités, les tâches et en règle générale les objets de l'environnement
\item Définir une fonction de coût
\item Définir un algorithme d'affectation
\end{itemize}

Une stratégie peut être vu comme un algorithme dont on va pouvoir changer les briques évoquées ci-dessus.\\\\

Cela fait apparaître plusieurs axes de travail qui seront autant de modules du projet :
\begin{itemize}
\item L'IA et l'environnement
\item Un module d'algorithmes d'indexation spatiale et  de partitionnement spatial
\item Un module d'algorithmes de plus court chemin (composant principal d'une fonction de coût)
\item Un modèle d'algorithmes d'affectation
\end{itemize}

On pourrait imaginer un module dédié aux méthodes d'apprentissage et au modèle d'observation mais étant donné le temps à consacrer au projet, on choisira un modèle simple, déterministe et omniscient (c'est à dire que l'on aura pas d'apprentissage, que l'on considérera avoir toujours prit la bonne décision et l'ensemble de l'environnement est connu à chaque instant par l'IA).

\subsection{Module d'IA et d'environnement}

Il s'agit du module de plus « haut niveau » qui doit donner une API simple permettant de décrire un environnement spatial et ses objets (unités, tâches, obstacles), les contraintes liées aux tâches et enfin de quoi concevoir des stratégies données à l'IA (qu'on pourra intervertir pour par exemple constater de leurs différents effets).\\\\

Il devrait également reposer sur une technique d'apprentissage (par exemple : Algorithmes d'apprentissage par renforcement)  et un modèle de prise de décision (par exemple : Processus de décision markovien partiellement observable) mais comme précisé plus haut, nous n'aurons jamais le temps d'envisager divers modèles de la sorte.

\subsection{Module d'indexation spatiale et de partitionnement spatial}

Ce module devra proposer deux choses :
\begin{itemize}
\item Des structures de données permettant l'indexation et le partitionnement spatial
\item Des algorithmes permettant de réaliser l'indexation et le partitionnement
\end{itemize}

Il s'agira de proposer une API la plus unifiée possible pour permettre de changer de structures de données indépendamment de l'algorithme afin de pouvoir comparer leurs performances par exemple.\\\\

On peut citer quelques structures de données qui pourront être étudiées : les Arbre kd, R Tree, Hilbert R Tree, BSP, QuadTree...\\\\

L'objectif de ce module est d'avoir des algorithmes qui vont permettre à l'IA de transformer un environnement « brut », à savoir un plan avec une liste d'objets dessus, en une ou plusieurs structures plus intelligente pour réduire la complexité de certaines heuristiques de plus court chemin ou d'affectation.

\subsection{Module d'algorithmes de plus court chemin}

Étant donné que nous travaillons avec une dimension spatiale, les fonctions de coûts d'affectation d'une unité à une tâche vont être majoritairement basées sur la distance à parcourir. De plus, il faudra ensuite que chaque unité puisse se déplacer vers la tâche qui lui a été affectée le plus rapidement.\\\\

L'objectif de ce module est donc de proposer divers algorithmes de plus court chemin, toujours avec une API la plus unifiée possible.

\subsection{Module d'affectation}

Il existe plusieurs approches pour l'affectation et ce module doit proposer plusieurs algorithmes pour résoudre le problème une fois l'évaluation des solutions possibles obtenues.\\
On peut citer l'algorithme de Kuhn mais également la recherche des voisins les plus proches où l'algorithme des axes principaux.

\subsection{Note sur l'implémentation et les objectifs}

Le projet étant ambitieux à la vue de la multitude des algorithmes existants tant pour le plus court chemin que pour l'indexation spatiale voire pour la description des contraintes et objectifs, le travail réalisé sera axé sur la modélisation et l'architecture du framework. L'implémentation sera partielle mais les bonnes pratiques de développement occuperont une part importante de celle-ci dans le but de valoriser le livrable et permettre une implémentation complète en dehors du cadre de ce projet et d'ajouter de nouveaux algorithmes facilement.\\\\

On choisira de n'implémenter qu'un algorithme de chaque module (pour pouvoir proposer une IA fonctionnelle). Notre choix s'est porté sur des algorithmes classiques et relativement simple d'implémentation :
\begin{itemize}
\item Partitionnement : Simple grille et QuadTree
\item Indexation : R-Tree
\item Plus Court Chemin : A*
\item Affectation : Méthode hongroise
\end{itemize}
