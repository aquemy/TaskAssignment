L'affectation est un problème d’optimisation combinatoire qui consiste, dans sa version simple, à affecter un certain nombre de ressources disponibles à un certain nombre de tâches dans l'objectif d'optimiser une fonction objectif. Il peut s'agir de minimiser des coûts ou maximiser les bénéfices. Ce problème peut être résolu en temps polynomial par la méthode hongroise également appelée algorithme de Kunh.\\
On peut étendre ce problème par des objectifs multiples, des contraintes changeantes en temps réel ou encore des contraintes probabilités traduisant une observation ou une connaissance partielle de l'environnement. On peut alors parler plus largement de planification, qui est un domaine ouvert de l'intelligence artificielle où de nombreuses formulations font apparaître des problèmes de la classe NP-difficile ou NP-complet sur lesquels de nombreuses équipes de recherches travaillent.\\\\

Dans le cadre de notre projet C++ nous avons choisi de réaliser un (début de) framework permettant la modélisation et l'implémentation de problème de planification spatiale sous contraintes. Il s'agira de proposer des briques logicielles permettant la création et la gestion d'IA pour la résolution de ce problème.\\
Enfin, pour illustrer la résolution temps réel des problèmes, en plus du framework, une application de simulation d'une Intelligence Artificielle sera développée. Il s'agira d'une application graphique montrant des unités d'un jeu vidéo affectées à divers postes pour maximiser divers objectifs pouvant changer au court du temps (équilibrage de ressources, objectif de construction, par exemple).
